\documentclass[12pts]{article}
\begin{document}

HTML = Hyper Text Markup Language (can turn text into links, tables, animations
       etc...)//
CSS: Cascading Style Sheets (makes the webpages look good)
//

SOME SYNTAX://

1) Always put <!DOCTYPE html> on the first line. This tells the browser what language it's reading (in this case, HTML).//

2) Always put <html> on the next line. This starts the HTML document.//

3) Always put <\slash html> on the last line. This ends the HTML document.//

4) Things inside <> are called tags. They usually come in pairs, an opening (<>)
   and a closing (<\slash> tag.//
   Tags nest, so they must be closed in the right order.//

5) Two parts in html file: The head (info about the file, and the title).//
   And the body (where the content is put)//
   <head> <\slash head>; <title> <\slash title>; <body> <\slash body>//

6) Inside the body, there are paragraphs <p> <\slash p>//

7) Headings for paragraphs are indicated by <h1> <\slash h1>. There are 6 headings size, from big (<h1>) to small (<h6>).//

8) hyperlink: oopening <a> tag, but there is an attribute to the tag, which is\\
   href = "www.whatever site.com".//
   so open hyperlink with\\ 
   <a href="www.whatever site.com">\\
   then add some description//
   then close the link <\slash a>//

9) Add image with the <img> tag, with the attribute src (source). But here
   there is no closing tag, just add / at the end of the opening tag. eg.//
   <img src="url" \slash>//

10) To make an image a link: <a href = www.> <img src=" " \slash> <\slash a>//

11) Ordered list with <ol> </ol>. Each item in the list starts with
    <li> </li>//
    Unordered lists are with <ul> </ul>

12) Comment tag: <!--   -->//

13) tags like <p> can have attributes, such as style="font-size: 12px"//

14) combining attributes://
    <h2 style="color: green; font-size:12px">//

15) modifying background color://
<p style="background-color: red;">Hello!</p>//

16) bold text: <strong>      <\slash strong>//
    italics: <em> </em>//

17) Tables: <table> </table>//
    New row: <tr> </tr>//
    New cell: <td> </td>//
    For a better table look: <thead></thead> with <th></th>//
    and <tbody></tbody>//
    Table title spanning several columns, attribute colspan=""//

18) Using several attribute://
<th colspan="2"style="color:red">Famous Monsters by Birth Year</th> (no comma)//

19) Divideing the pages in units: <div></div>

20) To work on the style of some elements in the text, use <span attributes> </span>//



/end{document}
